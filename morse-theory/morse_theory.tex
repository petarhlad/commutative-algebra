%%%%%%%%%%%%%%%%%%%%%%%%%%%%%%%%%%%%%%%%%
% Short Sectioned Assignment
% LaTeX Template
% Version 1.0 (5/5/12)
%
% This template has been downloaded from:
% http://www.LaTeXTemplates.com
%
% Original author:
% Frits Wenneker (http://www.howtotex.com)
%
% License:
% CC BY-NC-SA 3.0 (http://creativecommons.org/licenses/by-nc-sa/3.0/)
%
%%%%%%%%%%%%%%%%%%%%%%%%%%%%%%%%%%%%%%%%%

%----------------------------------------------------------------------------------------
%	PACKAGES AND OTHER DOCUMENT CONFIGURATIONS
%----------------------------------------------------------------------------------------

\documentclass[paper=a4, fontsize=11pt]{scrartcl} % A4 paper and 11pt font size

\usepackage[T1]{fontenc} % Use 8-bit encoding that has 256 glyphs
%\usepackage{fourier} % Use the Adobe Utopia font for the document - comment this line to return to the LaTeX default
\usepackage[english]{babel} % English language/hyphenation
\usepackage[utf8]{inputenc}  %allows non-English characters
\usepackage{amsmath,amsfonts,amsthm, amssymb} % Math packages
\usepackage{float}

\usepackage{sectsty} % Allows customizing section commands
%\allsectionsfont{\centering \normalfont\scshape} % Make all sections centered, the default font and small caps
%\allsectionsfont{\centering}

\usepackage{fancyhdr} % Custom headers and footers
\pagestyle{fancyplain} % Makes all pages in the document conform to the custom headers and footers
\fancyhead{} % No page header - if you want one, create it in the same way as the footers below
\fancyfoot[L]{} % Empty left footer
\fancyfoot[C]{} % Empty center footer
\fancyfoot[R]{\thepage} % Page numbering for right footer
\renewcommand{\headrulewidth}{0pt} % Remove header underlines
\renewcommand{\footrulewidth}{0pt} % Remove footer underlines
\setlength{\headheight}{13.6pt} % Customize the height of the header

%\usepackage{geometry}
%\usepackage{pdflscape}


%\numberwithin{equation}{section} % Number equations within sections (i.e. 1.1, 1.2, 2.1, 2.2 instead of 1, 2, 3, 4)
%\numberwithin{figure}{section} % Number figures within sections (i.e. 1.1, 1.2, 2.1, 2.2 instead of 1, 2, 3, 4)
%\numberwithin{table}{section} % Number tables within sections (i.e. 1.1, 1.2, 2.1, 2.2 instead of 1, 2, 3, 4)

%\setlength\parindent{0pt} % Removes all indentation from paragraphs - comment this line for an assignment with lots of text

\usepackage{caption}
%\usepackage{topcapt}

\usepackage{booktabs}

\usepackage{graphicx}
\usepackage{adjustbox}


%shortcuts for typing variance and expectation
\newcommand{\E}{\mathrm{E}}
\newcommand{\Var}{\mathrm{Var}}
\newcommand{\RR}{{\mathbb R}}

\theoremstyle{plain}
\newtheorem{thm}{Theorem}[section]
\newtheorem{cor}[thm]{Corollary}

\theoremstyle{definition}
\newtheorem{prob}[thm]{Problem}
\newtheorem{defn}[thm]{Definition} 
\newtheorem{exmp}[thm]{Example}

\usepackage{listings}
\usepackage{hyperref}

%----------------------------------------------------------------------------------------
%	TITLE SECTION
%----------------------------------------------------------------------------------------

\newcommand{\horrule}[1]{\rule{\linewidth}{#1}} % Create horizontal rule command with 1 argument of height

\title{
\normalfont \normalsize
\textsc{UPC - Commutative Algebra} \\ [25pt] % Your university, school and/or department name(s)
\horrule{0.5pt} \\[0.4cm] % Thin top horizontal rule
\huge Discrete Morse Theory \\ % The assignment title
\horrule{2pt} \\[0.5cm] % Thick bottom horizontal rule
}

\author{Petar Hlad Colic} % Your name

\date{\normalsize\today} % Today's date or a custom date

\begin{document}


\maketitle % Print the title


%----------------------------------------------------------------------------------------
%	INTRO
%----------------------------------------------------------------------------------------

\section{Introduction}
%Write here what will do 


\section{Monomial Ideals}
Monomial ideals are those ideals generated by monomials. For each monomial ideal there exists a unique minimal generating set of monomials.

\begin{defn}
Let $G = (V(G),E(G))$ be a finite simple graph. The \textit{edge ideal} associated to $G$ is the monomial ideal
$$ I(G) = \langle x_i x_j \vert \lbrace	x_i, x_j \rbrace \in E(G) \rangle \subseteq R = k \lbrack x_1, \dots, x_n \rbrack$$

The \textit{cover ideal} is the monomial ideal
$$J(G) = \bigcap_{\lbrace	x_i, x_j \rbrace \in E(G)}  \langle x_i, x_j \rangle \subseteq R = k \lbrack x_1, \dots, x_n \rbrack$$

\end{defn}

As edge ideals are square-free, we can apply the theory of Stanley Reissner ideals and simplicial complexes. \cite{Mo12}

\begin{defn}
A simplicial complex on $V = \lbrace x_1, \dots , x_n \rbrace$ is a subset $\Delta$ of the power set of $V$ (i.e. $\Delta \subseteq \mathcal{P}(V)$) such that
\renewcommand{\labelenumi}{(\roman{enumi})}
\begin{enumerate}
\item if $F \in \Delta$, and $G\subseteq F$, them $G \in \Delta$.
\item $\lbrace x_i \rbrace \in \Delta$ for all $i$.
\end{enumerate}
\end{defn}

\begin{defn}
The \textit{Stanley-Reisner ideal} associated to a simplicial complex $\Delta$ is the square-free monomial ideal $$I_{\Delta} = \langle x_{W} \vert W \notin \Delta \rangle$$
Where $x_{W} := \prod_{x_i \in W} x_i$, for $W \subseteq V(G)$.
\end{defn}

For any simplicial complex $\Delta$, the primary decomposition of $I_{\Delta}$ can be described in terms of the \textit{facets} of $\Delta$. Recall that $F \in \Delta$ is a facet if $F$ is a face that is maximal under inclusion. For each facet $F$, let $$P_{F} = \langle x_i \vert x_i \notin F \rangle$$

\begin{thm} (\cite[Theorem 3.1.34]{Mo12}) Let $\Delta$ be a simplicial complex with facets $F_1, \dots, F_t$. Then $$I_{\Delta} = P_{F_1} \cap P_{F_2} \cap \cdots \cap P_{F_t}$$
\end{thm}

\begin{defn} A subset $W \subseteq V(G)$ is a \textit{vertex cover} if $W \cap e \neq \emptyset$ for all $e \in E(G)$. A vertex cover $W$ is a \textit{minimal vertex cover} if no proper subset of $W$ is a vertex cover.
\end{defn}

\begin{cor} (\cite[Corollary 3.1.35]{Mo12})
Let $W_1, \dots, W_t$ be the minimal vertex covers of $G$, and set $\langle W_i \rangle = \langle x_j \vert x_j \in W_i \rangle$. Then
$$I(G) = \langle W_1 \rangle \cap \dots \cap \langle W_t \rangle$$
\end{cor}

\begin{defn}
Let $I$ be a square-free monomial ideal with primary decomosition $$I = \langle x_{1,1}, x_{1,2}, \dots, x_{1,s_1} \rangle \cap \langle x_{2,1}, x_{2,2}, \dots, x_{2,s_2} \rangle \cap \dots \cap \langle x_{t,1}, x_{t,2}, \dots, x_{t,s_t} \rangle$$

The \textit{Alexander Dual} of $I$, denoted $I^{\vee}$, is the square-free monomial ideal $$I^{\vee} = \langle x_{1,1} x_{1,2} \cdots x_{1,s_1}, x_{2,1} x_{2,2} \cdots x_{2,s_2}, \dots, x_{t,1} x_{t,2} \cdots x_{t,s_t} \rangle$$
\end{defn}

\begin{cor} (\cite[Corollary 3.1.38]{Mo12})
Let $G$ be a graph. Then $I(G)^{\vee} = J(G)$.
\end{cor}

%%%%%%%%%%%%%%%%%%%%%%%%%%%%%%%%%%%%%%%%%%%%%%%%%%%%%%%%%%%%%%%%
\section{Free Resolutions}
\begin{defn}
If $I \subseteq R$ an ideal, then a \textit{free resolution} of $R/I$ is an exact sequence $$\mathbb{F} : \dots \xrightarrow{\phi_{n}} F_n \xrightarrow{\phi_{n-1}} F_{n-1} \dots \xrightarrow{\phi_{0}} F_0 \rightarrow R/I \rightarrow 0$$ where each of the of the $F_i$ is a free $R$-module.

$$F_i = \bigoplus_{\alpha \in \mathbb{Z}^n} R(-\alpha)^{\beta_{i,\alpha}}$$

We say that $\mathbb{F}$ is \textit{minimal} if each of the modules $F_i$ has minimum possible rank; in this case the ranks are the \textit{Betti numbers} of $S/I$ and these form a set of invariants.
\end{defn}

\subsection{Cellular Resolutions}
A CW-complex $X$ is a topological space obtained by attaching cells of increasing dimensions to a discrete set of points $X^{(0)}$. Let $X^{(i))}$ denote the set of $i$-cells of $X$ and consider the set of all cells $X^{(\ast)} := \bigcup_{i\geq 0} X^{(i)}$. Then we can view $X^{(\ast)}$ as a poset with the partial order given by $\sigma ' \leq sigma$ if and only if $\sigma '$ is contained in the closure of $\sigma$. We can also give a $\mathbb{Z}^n$-graded structure to $X$ by means of an order preserving map $gr: X^{(\ast)} \longrightarrow \mathbb{Z}^{n}_{\geq 0}$.

We say that the free resolution is \textit{cellular} (or is a \textit{CW-resolution}) if there exusts a $\mathbb{Z}^n$-graded CW-complex $(X,gr)$ such that, for all $i \geq 1$:

\begin{itemize}
\item there exists a basis $\lbrace e_{\sigma} \rbrace$ of $F_i$ indexed by the $(i-1)$-cells of $X$, such that if $e_{\sigma} \in R(-\alpha)^{\beta_{i,\alpha}}$ then $gr(\sigma) = \alpha$, and
\item the differential $d_i: F_i \longrightarrow F_{i-1}$ is given by $$e_{\sigma} \mapsto \sum_{\sigma \geq \sigma ' \in X^{(i-1)}} \lbrack \sigma : \sigma' \rbrack \mathbf{x}^{gr(\sigma) - gr(\sigma ')} e_{\sigma '}, \quad \forall \sigma \in X^{(i)}$$

where $\lbrack \sigma : \sigma' \rbrack$ denotes the coefficient of $\sigma '$ in the image of $\sigma$ by the differential map in the cellular homology of $X$.
\end{itemize}

From now on, we will the denote the free resolution as $\mathbb{F}_{\bullet} = \mathbb{F}_{\bullet}^{(X,gr)}$. If $X$ is a simplicial complex, we say that the free resolution is \textit{simplicial}.
\subsubsection{Taylor Resolution}

\subsubsection{Scarf Complex}


\subsubsection{Lyubeznik Resolution}

%%%%%%%%%%%%%%%%%%%%%%%%%%%%%%%%%%%%%%%%%%%%%%%%%%%%%%%%%%%%%%%%
\section{Results}
%Write here what the result is
In figure~\ref{fig:gale} we show all neighbourly Gale diagrams we found.
\begin{figure}[!htb]
\centering
%\includegraphics[trim={0 5cm 0 5cm},clip,width=\textwidth]{}
\caption{All neighbourly Gale Diagrams, some isomorphic.}
\label{fig:gale}
\end{figure}\\

\begin{figure}[!htb]
%\lstinputlisting{polytopes.txt}
\caption{The coordinates for all $3$ different neighbourly $4$-polytopes on $8$ vertices up to combinatorial equivalence}
\label{fig:polyPoints}
\end{figure}

(\cite{BaWe02})

\begin{thebibliography}{9}
\bibitem{BaWe02}
E. Batzies, V. Welker, \textit{Discrete Morse theory for cellular resolutions}. Journal f{\"u}r die reine und angewandte Mathematik \textbf{543} (2002), 147-168.
\bibitem{Ch00}
Manoj K. Chari, \textit{On discrete Morse functions and combinatorial decompositions}. Discrete Mathematics \textbf{217}, (2000).
\bibitem{JoWe}
M. J{\"o}llenbeck, V. Welker, \textit{Minimal Resolutions via Algebraic Discrete Morse Theory}, (2005).
\bibitem{AlFeGi17}
J. \`Alvarez Montaner, O. Fern\'andez-Ramos, Ph. Gimenez, \textit{Pruned cellular free resolutions of monomial ideals}, (2017).
\bibitem{Me11}
J. Mermin, \textit{Three simplicial resolutions}, (2011).
\bibitem{Mo12}
A.M. Bigatti, P. Gimenez, E. S\`aenz-de-Cabez\'on, Eds., \textit{Monomial Ideals, Computations and Applications}, (2012).
\end{thebibliography}
\end{document}